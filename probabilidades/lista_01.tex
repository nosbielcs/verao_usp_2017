\documentclass[12pt]{article}
\usepackage[brazilian]{babel} %traduz para o pt-br
\usepackage[utf8]{inputenc} %reconhece acentua\c{c}\~{a}o
\usepackage{lipsum} %gera textos aleat\'{o}rios
\usepackage{graphicx} %uso de imagens
\usepackage{amssymb} % para exibir simbolos matematicos
\usepackage{subfig} %uso de caixa de figuras
\usepackage{amsmath} %para usar begin{cases}
\usepackage{amsfonts} %para usar mathbb
\usepackage{float} %fixa tabela e imagens flutuantes
\graphicspath{{images/}} %diretorio onde ficar\~{a}o imagens
\usepackage{latexsym}
\usepackage{lineno} %para mostrar numeração de linhas

%opening
\title{Lista 01}
\author{Cleibson Aparecido de Almeida}

\begin{document}

\maketitle

\section*{Exercício 03}
Dados $A, B_1, ... , B_n \leq 1$, subconjuntos de $\mathcal{U} = \mathbb{Z}$, mostre que:

a) $(A^c)^c = A$

Seja: $A^c = 1-A$;

Então: $(1-A)^c = (1-(1-A)) = 1-1+A = A$

\bigskip

b) $(\overset{n}{\underset{i=1}{\cup}} B_i)^c = \overset{n}{\underset{i=1}{\cap}} B_i^c$ e $(\overset{n}{\underset{i=1}{\cap}} B_i)^c = \overset{n}{\underset{i=1}{\cup}} B_i^c$

Seja: $(\overset{n}{\underset{i=1}{\cup}} B_i)^c$

$= (B_1 \cup B_2 \cup ... \cup B_n)^c$

$= B_1^c \cup B_2^c \cup ... \cup B_n^c$

$= B_1 \cap B_2 \cap ... \cap B_n$

$= \overset{n}{\underset{i=1}{\cap}} B_i^c$

e seja: $(\overset{n}{\underset{i=1}{\cap}} B_i)^c$

$= (B_1 \cap B_2 \cap ... \cap B_n)^c$

$= B_1^c \cap B_2^c \cap ... \cap B_n^c$

$= B_1 \cup B_2 \cup ... \cup B_n$

$= \overset{n}{\underset{i=1}{\cup}} B_i^c$

\bigskip

c) Se $\mathcal{B}=\{B_1,...,B_n\}$ é uma partição de $\mathcal{U}$, então a coleção $\{B_1 \cap A, ..., B_n \cap A\}$ é uma partição de A.

Aplicando a propriedade $(\underset{i \in I}{\cup A_i}) \cup B = \underset{i \in I}{\cup} (A_i \cup B)$ ao problema, temos que:

$(\underset{i \in I}{\cup B_i}) \cup A$

$= \underset{i \in I}{\cup} (B_i \cup A)$

$=(B_1 \cap A) \cup (B_2 \cap A) \cup ... \cup (B_n \cap A)$

$=\{B_1 \cap A, B_2 \cap A, ... , B_n \cap A\}$

\end{document}
