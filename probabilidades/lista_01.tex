\documentclass[12pt]{article}
\usepackage[brazilian]{babel} %traduz para o pt-br
\usepackage[utf8]{inputenc} %reconhece acentua\c{c}\~{a}o
\usepackage{lipsum} %gera textos aleat\'{o}rios
\usepackage{graphicx} %uso de imagens
\usepackage{amssymb} % para exibir simbolos matematicos
\usepackage{subfig} %uso de caixa de figuras
\usepackage{amsmath} %para usar begin{cases}
\usepackage{amsfonts} %para usar mathbb
\usepackage{float} %fixa tabela e imagens flutuantes
\graphicspath{{images/}} %diretorio onde ficar\~{a}o imagens
\usepackage{latexsym}
\usepackage{lineno} %para mostrar numeração de linhas

%opening
\title{Análise Combinatória, Probabilidades e Aplicações - Lista 01}
\author{{Arthur C. Leite}, {Cleibson Aparecido de Almeida}}

\begin{document}

\maketitle

\section*{Exercício 03}

Dados $A, B_1, ... , B_n \geq 1$, subconjuntos de $\mathcal{U} = \mathbb{Z}$, mostre que:

a) $(A^c)^c = A$

Seja: $A^c = 1-A$;

Então: $(1-A)^c = (1-(1-A)) = 1-1+A = A$

\bigskip

b) $(\overset{n}{\underset{i=1}{\cup}} B_i)^c = \overset{n}{\underset{i=1}{\cap}} B_i^c$ e $(\overset{n}{\underset{i=1}{\cap}} B_i)^c = \overset{n}{\underset{i=1}{\cup}} B_i^c$

Seja: $(\overset{n}{\underset{i=1}{\cup}} B_i)^c$

$= (B_1 \cup B_2 \cup ... \cup B_n)^c$

$= B_1^c \cup B_2^c \cup ... \cup B_n^c$

$= B_1 \cap B_2 \cap ... \cap B_n$

$= \overset{n}{\underset{i=1}{\cap}} B_i^c$

e seja: $(\overset{n}{\underset{i=1}{\cap}} B_i)^c$

$= (B_1 \cap B_2 \cap ... \cap B_n)^c$

$= B_1^c \cap B_2^c \cap ... \cap B_n^c$

$= B_1 \cup B_2 \cup ... \cup B_n$

$= \overset{n}{\underset{i=1}{\cup}} B_i^c$

\bigskip

c) Se $\mathcal{B}=\{B_1,...,B_n\}$ é uma partição de $\mathcal{U}$, então a coleção $\{B_1 \cap A, ..., B_n \cap A\}$ é uma partição de A.

Aplicando a propriedade $(\underset{i \in I}{\cup A_i}) \cup B = \underset{i \in I}{\cup} (A_i \cup B)$ ao problema, temos que:

$(\underset{i \in I}{\cup B_i}) \cup A$

$= \underset{i \in I}{\cup} (B_i \cup A)$

$=(B_1 \cap A) \cup (B_2 \cap A) \cup ... \cup (B_n \cap A)$

$=\{B_1 \cap A, B_2 \cap A, ... , B_n \cap A\}$

\section*{Exercício 09}

De quantas maneiras podemos distribuir n objetos em duas caixas de modo que nenhuma caixa fique vazia, quando:

a) Os objetos e as caixas são diferentes?

b) Os objetos são iguais e as caixas diferentes?

\section*{Exercício 11}

a) Um químico possui 10 tipos de substâncias: $/{A_1, A_2, .., A_n/}$. De quantos modos poderá combinar 6 dessas substâncias se, entre as dez, duas não podem estar juntas?

b) O mesmo químico tem a hipótese de que ao dissolver 5 doses de 2 ml das substâncias $/{A_1, .., A_10/}$ (as doses podem ser repetidas) em 5 ml de água, obterá uma solução útil ao combate da dengue. O químico precisa fazer um experimento no laboratório com todas as soluções possíveis. Qualé o número máximo de testes a serem feitos pelo químico? (suponha que a ordem de dissolução não afeta a solução final).

\section*{Exercício 13}

Seja $\mathcal{F}$ a classe das funções que associam o conjunto $\{1, 2, ..., 2n+1\}$ ao conjunto $\{1,2, ... , 2n\}$, $n \geq 1$, isto é:

$\mathcal{F} = \{ f: \{1, 2, ...,2n+1 \} \rightarrow \{1, 2, ..., 2n \} \}$

Sejam ainda os seguintes subconjuntos de $\mathcal{F}$:

$\mathcal{I}:$ constituído pelas funções de $\mathcal{F}$ que associam a cada número ímpar um número par,

$\mathcal{S}:$ constituído pelas funções sobrejetoras de $\mathcal{F}$.

Determine $|\mathcal{F}|, |\mathcal{I}| e |\mathcal{S}|$.

\section*{Exercício 15}

Determine os números de possíveis anagramas das palavras SUSSURRO, VESTIBULAR e BATATA.

a) SUSSURRO

b) VESTIBULAR

c) BATATA

\section*{Exercício 20}

Quantas são as soluções não negativas da inequação $x + y + z \leq 2$?

\end{document}
