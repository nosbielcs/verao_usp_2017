\documentclass[12pt]{article}
\usepackage[brazilian]{babel} %traduz para o pt-br
\usepackage[utf8]{inputenc} %reconhece acentua\c{c}\~{a}o
\usepackage{lipsum} %gera textos aleat\'{o}rios
\usepackage{graphicx} %uso de imagens
\usepackage{amssymb} % para exibir simbolos matematicos
\usepackage{subfig} %uso de caixa de figuras
\usepackage{amsmath} %para usar begin{cases}
\usepackage{amsfonts} %para usar mathbb
\usepackage{float} %fixa tabela e imagens flutuantes
\graphicspath{{images/}} %diretorio onde ficar\~{a}o imagens
\usepackage{latexsym}
\usepackage{lineno} %para mostrar numeração de linhas

%opening
\title{Análise Combinatória, Probabilidades e Aplicações - Lista 01}
\author{{Arthur Cardoso Leite}, {Cleibson Aparecido de Almeida}}

\begin{document}

\maketitle

\section*{Exercício 03}

Dados $A, B_1, ... , B_n \geq 1$, subconjuntos de $\mathcal{U} = \mathbb{Z}$, mostre que:

a) $(A^c)^c = A$

Aplicando a propriedades 9 [pág.15], em \cite{morgado_1991}. 

Seja: $A^c = 1-A$;

Então: $(1-A)^c = (1-(1-A)) = 1-1+A = A$

\bigskip

b) $(\overset{n}{\underset{i=1}{\cup}} B_i)^c = \overset{n}{\underset{i=1}{\cap}} B_i^c$ e $(\overset{n}{\underset{i=1}{\cap}} B_i)^c = \overset{n}{\underset{i=1}{\cup}} B_i^c$

Seja: $(\overset{n}{\underset{i=1}{\cup}} B_i)^c$

Aplicando as propriedades 10 e 11 [pág.15], em \cite{morgado_1991}. 

$= (B_1 \cup B_2 \cup ... \cup B_n)^c$

$= B_1^c \cap B_2^c \cap ... \cap B_n^c$

$= \overset{n}{\underset{i=1}{\cap}} B_i^c$

e seja: $(\overset{n}{\underset{i=1}{\cap}} B_i)^c$

$= (B_1 \cap B_2 \cap ... \cap B_n)^c$

$= B_1^c \cup B_2^c \cup ... \cup B_n^c$

$= \overset{n}{\underset{i=1}{\cup}} B_i^c$

\bigskip

c) Se $\mathcal{B}=\{B_1,...,B_n\}$ é uma partição de $\mathcal{U}$, então a coleção $\{B_1 \cap A, ..., B_n \cap A\}$ é uma partição de A.

Aplicando a propriedade $(\underset{i \in I}{\cup A_i}) \cup B = \underset{i \in I}{\cup} (A_i \cup B)$ ao problema, temos que:

$(\underset{i \in I}{\cup B_i}) \cup A$

$= \underset{i \in I}{\cup} (B_i \cup A)$

$=(B_1 \cap A) \cup (B_2 \cap A) \cup ... \cup (B_n \cap A)$

$=\{B_1 \cap A, B_2 \cap A, ... , B_n \cap A\}$

\section*{Exercício 09}

De quantas maneiras podemos distribuir n objetos em duas caixas de modo que nenhuma caixa fique vazia, quando:

\bigskip

a) Os objetos e as caixas são diferentes?

Quando os objetos e caixas são diferentes, existindo n objetos, tem, dessa forma, $2^n$ modos de arranjá-los em duas caixas distintas.

Porém, excluí-se duas dessas opções, uma vez que nenhuma das duas caixas pode ficar vazia. Dessa forma, têm-se como resposta do exercício $2^n-1$ modos de distribuir n objetos distintos em duas caixas distintas. 

\bigskip

b) Os objetos são iguais e as caixas diferentes?

Considerando as caixas A e B, têm-se:

\begin{tabular}{cc}
	$A$ & $B$ \\ 
	**** ... nX * & I \\ 
\end{tabular} 
   
Utilizando-se o conceito de Combinações Completas, têm-se: \medskip

$\displaystyle \dfrac{(n+1)!}{n!}=n+1$ \medskip

Uma vez que as caixas devem ter no mínimo um objeto, deve-se desconsiderar dois arranjos. Assim sendo, a resposta do problema é $n+1-2=n-1$. Portanto, $n-1$ modos de arranjar os objetos.

\section*{Exercício 11}

a) Um químico possui 10 tipos de substâncias: $/{A_1, A_2, .., A_n/}$. De quantos modos poderá combinar 6 dessas substâncias se, entre as dez, duas não podem estar juntas?

\textbf{1º Caso:} $A_1$ está na reação
Arranjemos do seguinte modo:
$A_1 \times 8 \times 7 \times 6 \times 5 \times 4 = 6720$.

Como a ordem não importa neste caso, dividiremos o resultado por 5!, resultando em 56 modos distintos.

\textbf{2º Caso:} $A_2$ está na reação
Arranjemos do seguinte modo:
$A_2 \times 8 \times 7 \times 6 \times 5 \times 4 = 6720$.

Como a ordem não importa neste caso, dividiremos o resultado por 5!, resultando em 56 modos distintos.

\textbf{3º Caso:} Nem $A_1$, nem $A_2$ estão na reação. Arranjemos do seguinte modo:
$8 \times 7 \times 6 \times 5 \times 4 \times 3 = 20160$.

Como a ordem não importa neste caso, dividiremos o resultado por 6!, resultando em 28 modos distintos.

Assim sento, pelo Princípio Aditivo, têm-se $56 + 56 + 28 = 140$ reações distintas.

b) O mesmo químico tem a hipótese de que ao dissolver 5 doses de 2 ml das substâncias $/{A_1, .., A_10/}$ (as doses podem ser repetidas) em 5 ml de água, obterá uma solução útil ao combate da dengue. O químico precisa fazer um experimento no laboratório com todas as soluções possíveis. Qualé o número máximo de testes a serem feitos pelo químico? (suponha que a ordem de dissolução não afeta a solução final).

Como a ordem das substâncias não importa, temos pelo princípio da Combinação Completa a seguinte resolução:

\begin{tabular}{cccccccccc}
	$A_1$ & $A_2$ & $A_3$ & $A_4$ & $A_5$ & $A_6$ & $A_7$ & $A_8$ & $A_9$ & $A_10$ \\ 
	***** & I & I & I & I & I & I & I & I & I \\ 
\end{tabular}  
       
Assim sendo, temos $\displaystyle \dfrac{14!}{5! \times 9!} = 2002$. Logo, há 2002 testes a serem feitos.

\section*{Exercício 13}

Seja $\mathcal{F}$ a classe das funções que associam o conjunto $\{1, 2, ..., 2n+1\}$ ao conjunto $\{1,2, ... , 2n\}$, $n \geq 1$, isto é:

$\mathcal{F} = \{ f: \{1, 2, ...,2n+1 \} \rightarrow \{1, 2, ..., 2n \} \}$

Sejam ainda os seguintes subconjuntos de $\mathcal{F}$:

$\mathcal{I}:$ constituído pelas funções de $\mathcal{F}$ que associam a cada número ímpar um número par,

$\mathcal{S}:$ constituído pelas funções sobrejetoras de $\mathcal{F}$.

Determine $|\mathcal{F}|, |\mathcal{I}| e |\mathcal{S}|$.

\section*{Exercício 15}

Determine os números de possíveis anagramas das palavras SUSSURRO, VESTIBULAR e BATATA.

\bigskip

a) SUSSURRO - Esta palavra possui repetição de letras,portanto será aplicada a regra da \textit{permutação com elementos nem todos distintos}.

Temos então a seguinte organização das 8 letras: SSS UU RR O (3 S, 2 U, 2 R e 1 O), e com isso a fórmula será $\displaystyle P_{8}^{3221} $

$= C_8^3 \times C_5^2 \times C_3^2 \times C_1^1$

$=56 \times 10 \times 3 \times 1 = 1680$ anagramas

\bigskip

b) VESTIBULAR - Esta palavra não possui letras repetidas,portanto trata-se de uma \textit{permutação simples}.

Temos então $P_{10}$

$= 10!$

$= 10 \times 9 \times ... \times 1$

$= 3628800$ anagramas

\bigskip

c) BATATA - Esta palavra possui repetição de letras,portanto será aplicada a regra da \textit{permutação com elementos nem todos distintos}.

Temos então a seguinte organização das 6 letras: AAA TT B (3 A, 2 T e 1 B), e com isso a fórmula será $\displaystyle P_{6}^{321} $

$= C_6^3 \times C_3^2 \times C_1^1$

$= 20 \times 3 \times 1 = 60$ anagramas

\section*{Exercício 20}

Quantas são as soluções não negativas da inequação $x + y + z \leq 2$?

Trata-se de um problema envolvendo \textbf{combinações completas}. \medskip

$CR_n^p = C_{n+p-1}^p$ \medskip

E sendo $\displaystyle C_{n+p-1}^p = \binom{n+p-1}{p} = \frac{(n+p-1)!}{(n-1)! \times p!}$

Então, são apresentadas todas as possibilidades de solução da inequação:

$x + y + z = 2 \rightarrow CR_3^2 \rightarrow C_4^2$; aqui tem-se n=3 e p=2.

$x + y + z = 1 \rightarrow CR_3^1 \rightarrow C_3^1$; aqui tem-se n=3 e p=1.

$x + y + z = 0 \rightarrow CR_3^0 \rightarrow C_2^0$; aqui tem-se n=3 e p=0. \medskip

Aplicando o \textit{principio aditivo}, temos que as soluções não negativas para inequação pode ser dada por $C_4^2 + C_3^1 + C_2^0 = 6 + 3 + 1 = 10$ maneiras de solucionar.

\bibliographystyle{abbrv}
\bibliography{bibliography}

\end{document}
